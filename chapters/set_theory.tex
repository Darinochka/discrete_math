\chapter{Теория множеств}
\section{Основные понятия}

\textbf{Множество} - это совокупность каких-либо объектов. 
Множества бывают конечными, бесконечными пустыми, счестными и бесчетными.

Мощность множества: $|A|$

Если между двумя множествами $A$ и $B$ можно установить взаимно-однозначное соответствие,
тогда множества называются \textbf{равномощными}.
\begin{equation}
    |A| = |B|
\end{equation}

\textbf{Теорема Кантора}. Множество подмножеств любого множества, имеет мощность больше,
чем само множество.

Обычно в множество подмножеств входит пустое множество и само множество.

Чтобы проверить теорему Кантора, рассмотрим конечное множество и рассмотрим множество
подмножеств конечного множества из n элементов. Например, рассмотрим множество из 2-х
элементов 
\begin{equation*}
    A = \{1, 2\} 
\end{equation*}

Множество подмножеств равно
\begin{equation*}
    A_2 = \{\emptyset, \{1\}, \{2\}, \{1, 2\}\}
\end{equation*}

Рассмотрим из множество 3-х элементов
\begin{equation*}
    B = \{1, 2, 3\} 
\end{equation*}

Множество подмножеств равно
\begin{equation*}
    B_3 = \{\emptyset, \{1\}, \{2\}, \{3\}, \{1, 2\}, \{1, 3\}, \{2, 3\}, \{1, 2, 3\}\}
\end{equation*}

Напрашивается закономерность, что
\begin{equation}
    |A_n| = 2^{|A|}
\end{equation}
где $|A_n|$ - мощность множества подможеств множества из n элементов

Выведем так же несколько формул для мощность множества подмножеств:
\begin{gather}
    |A_n| = |A_{n-1}| + |A_{n-1}| = 2|A_{n-1}|\\
    |A_n| = 2|A_{n-1}| = 2^{2}|A_{n-1}| = ... = 2^{n}|A_0| = 2^{n}
\end{gather}

Поставим вопрос: сколько будет подмножеств из n элементов содержащих k элементов? Отвечаю.
$C^{k}_n$ - число подмножеств из k элементов.

Стоит заметить, что
\begin{equation}
    \sum_{k=0}^{n} C^{k}_n = 2^{n}
\end{equation}

\section{Операции над множествами}

Рассмотрим основные операции над множествами.

\textbf{Объединение} 
\begin{equation}
    A \cup B = \{x: x \in A \cup x \in B\}
\end{equation}

\textbf{Пересечение} 
\begin{equation}
    A \cap B = \{x: x \in A \cap x \in B\}
\end{equation}

\textbf{Дополнение} 
\begin{equation}
    \overline A = \{x: x \notin A\}
\end{equation}

\textbf{Вычитание} 
\begin{equation}
    A \backslash B = \{x: x \in A \cap x \notin B\} = A \cap \overline B
\end{equation}

\textbf{Сумма} 
\begin{equation}
    A \oplus B = (A \backslash B) \cup (B \backslash A) = (A \cap \overline B) \cup (B \cap \overline A)
\end{equation}


\section{Законы теории множеств}

\begin{enumerate}
    \item \textbf{Коммутативность}
        \begin{gather}
            A \cup B = B \cup A \\
            A \cap B = B \cap A
        \end{gather}

    \item \textbf{Ассоциативность}
        \begin{gather}
            A \cup (B \cup C) = (A \cup B) \cup C \\
            A \cap (B \cap C) = (A \cap B) \cap C
        \end{gather}

    \item \textbf{Закон идентичности}
        \begin{gather}
            A \cup A = A \\
            A \cap A = \emptyset
        \end{gather}

    \item \textbf{Закон дистрибутивности}
        \begin{gather}
            A \cup (B \cap C) = (A \cup B) \cap (A \cup C)\\
            A \cap (B \cup C) = (A \cap B) \cup (A \cap C)
        \end{gather}

    \item \textbf{Закон поглощения}
        \begin{gather}
            A \cup (A \cap B) = A \\
            A \cap (A \cup B) = A
        \end{gather}

    \item \textbf{Закон де-Моргана}
        \begin{gather}
            \overline{A \cup B} = \overline A \cap \overline B \\
            \overline{A \cap B} = \overline A \cup \overline B
        \end{gather}

    \item \textbf{Закон нуля и единицы}
        \begin{gather}
            0 = \emptyset \\
            1 = \mathbb{U}
        \end{gather}

    \item \textbf{Двойное отрицание}
        \begin{equation}
            \overline{\overline A} = A
        \end{equation}

    \item \textbf{Законы склеивания}
        \begin{gather}
            (A \cup B) \cap (A \cup \overline B) = A \\
            (A \cap B) \cup (A \cap \overline B) = A 
        \end{gather}
\end{enumerate}

Пример некоторых доказательств.

\textbf{Закон поглощения}
\begin{gather}
    A \cup (A \cap B) = (A \cap 1) \cup (A \cap B) = A \cap (1 \cup B) = A \cap 1 = A \\
    A \cap (A \cup B) = (A \cup 0) \cap (A \cup B) = A \cup (0 \cap B) = A \cup 0 = A
\end{gather}

\textbf{Закон скеливания}
\begin{gather}
    (A \cup B) \cap (A \cup \overline B) = A \cup (B \cap \overline B) = A \cup 0 = A \\
    (A \cap B) \cup (A \cap \overline B) = A \cap (B \cup \overline B) = A \cap 1 = A
\end{gather}

Рассмотрим формулу дистрибутивности в следующем виде:
\begin{equation}
    A_1 \cap (B_1 \cap B_2 \cap ... \cap B_n) = (A_1 \cap B_1) \cup (A_1 \cap B_2) \cup ... \cup (A_1 \cap B_n)
\end{equation}

\textbf{Формула включения и исключения}
\begin{gather}
    |A \cup B| = |A| + |B| - |A \cap B| \\
    |A \cup B \cup C| = |A| + |B| + |C| - |A \cap B| - |A \cap C| - |B \cap C| + |A \cap B \cap C|
\end{gather}

Доказательства оставляю читателю :)

\begin{task}
    Сколько имеется целых чисел от 1 до 1000 которые не делятся на 3, 5, 7?

    \begin{solution}
        Возьмем $A_3$ - числа, делящиеся на 3, $A_5$ - числа, делящиеся на 5,
        $A_7$ - числа, делящиеся на 7.
    
        При этом, $|A_3| = \frac{1000}{3}$ = 333, $|A_5| = \frac{1000}{5} = 200$, $|A_7| = \frac{1000}{7} = 142$
        $|A_{15}| = \frac{1000}{3*5}$ = 66, $|A_{21}| = \frac{1000}{3*7}$ = 47, $|A_{35}| = \frac{1000}{7 * 5}$ = 28,
        $|A_{105}| = \frac{1000}{3*5*7}$ = 9
        \begin{equation*}
            |A_3 \cup A_5 \cup A_7| = 333 + 200 + 142 - 66 - 47 - 28 + 9 = 543
        \end{equation*}
        Ответ: $1000 - 543 = 457$
    \end{solution}
\end{task}
