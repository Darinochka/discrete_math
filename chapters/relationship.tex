\chapter{Отношения}
\section{Основные понятия}
Введем новые обозначения, аналогичные обозначениям в математической логике.


\begin{table}[h]
    \centering
    \begin{tabular}[c]{ | l | l | }
        \hline
        Отношения & Мат. логика \\ \hline
        $=$ & $\sim$   \\ \hline
        $\land$ & ,     \\ \hline
        $\lor$ &  ;    \\ \hline
        $\Rightarrow$ & $\to$ \\
        \hline
    \end{tabular}
\end{table}

Тавтология
\begin{equation}
    1 \to \overline P_1; \overline P_2; ... ; \overline P_{n-1}; \overline P_{n}; C
\end{equation}

Противоречие
\begin{equation}
    P_1, P_2, ..., P_{n-1}, P_n, \overline C \to 0
\end{equation}

\textbf{Отношения эквивалентности}
\begin{itemize}
    \item Рефлексивность
    \begin{equation}
        A \sim A
    \end{equation}

    \item Симметричность 
    \begin{equation}
        A \sim B \to B \sim A
    \end{equation}

    \item Транзитивность
    \begin{equation}
        A \sim B, B \sim C \to A \sim C
    \end{equation}
\end{itemize}

\textbf{Отношения порядка}
\begin{itemize}
    \item Рефлексивность
    \begin{equation}
        A \to A
    \end{equation}

    \item Антисимметричность
    \begin{equation}
        A \Rightarrow B, то \overline B \Rightarrow \overline A
    \end{equation}

    \item Транзитивность
    \begin{equation}
        A \Rightarrow B, B \Rightarrow C, \text{то } A \Rightarrow C
    \end{equation}
\end{itemize}

\section{Правила вывода клауз}
Выведем несколько формул, далее из них сформируем правила.

\begin{equation}\label{eq_clause}
    A \to B = \overline A \lor B = \overline A; B
\end{equation}

Теперь воспользуемся этой формулой, чтобы вывести другие:
\begin{gather}\label{eq_clauses}
    \begin{aligned}
        P_1, P_2, ..., P_n \Rightarrow C \\
        P_1, P_2, ..., P_n \to C \\
        \overline {P_1, P_2, ..., P_n} \lor C \\
        \overline P_1 \lor \overline P_2 \lor ... \lor \overline P_n \lor C \\
        \overline P_1 \lor \overline P_2 \lor ... \lor \overline P_{n-1} \lor (\overline P_n \lor C) \\
        \overline {P_1 \land P_2 \land ... \land P_{n-1}} \lor (\overline P_n \lor C) \\
        P_1, P_2, ..., P_{n-1} \Rightarrow (\overline P_n ; C) \\
        P_1, P_2, ..., P_{n-1} \Rightarrow \overline P_n; C
    \end{aligned}
\end{gather}

Так как далее пойдет немного сложноватая часть, я постараюсь пояснить.
\begin{enumerate}
    \item
        \begin{equation}\label{eq_1}
            A, A \to B \Rightarrow B
        \end{equation}
        Используем формулу \ref{eq_clause}, чтобы доказать
        \begin{gather*}
            A, \overline A; B \Rightarrow B \\
            \emptyset; B \Rightarrow B
        \end{gather*}
    \item
        \begin{equation}\label{eq_2}
            A \Rightarrow B \to A
        \end{equation}
        Так же заменяем $B \to A$ формулой \ref{eq_clause},
        а далее используем \ref{eq_clauses}, то есть из последнего получаем первое выражение
        \begin{gather*}
            A \Rightarrow \overline B; A \\
            A, B \Rightarrow A
        \end{gather*}
    \item
        \begin{equation}\label{eq_3}
            A \to B, B \to C \Rightarrow A \to C
        \end{equation}
        Используя предыдущую формулу \ref{eq_2} переносим $A$ налево
        \begin{equation*}
            A, A \to B, B \to C \Rightarrow C
        \end{equation*}
        Первую часть заменяем согласно \ref{eq_1} и снова используем эту же формулу.
        \begin{gather*}
            B, B \to C \Rightarrow C \\
            C \Rightarrow C
        \end{gather*}
        Доказано!
\end{enumerate}

\section{Задачи}
Теперь будем доказывать несколько примеров.

\begin{task}
    \begin{gather*}
        1 \Rightarrow (A \to B) \to ((C \to D) \to (A \lor C) \to (B \lor D)) \\
        A \to B, C \to D, A \lor C \Rightarrow B \lor D \\
        (A, A \to B) \lor (C, C \to D) \Rightarrow B \lor D
    \end{gather*}
\end{task}

\begin{task}
    \begin{gather*}
        A \to C, A \lor B, B \to D, D \to C \Rightarrow C \\
        A \to C, A \lor B, B \to C \Rightarrow C \\
        (A, A \to C); (B, B \to C) \Rightarrow C \\
        C \Rightarrow C
    \end{gather*}
\end{task}

\begin{task}
    \begin{gather*}
        (A \land B) \lor (C \land D); \overline A \Rightarrow C \\
        (A \land B) \lor (C \land D) \Rightarrow A; C \\
        A \lor C \Rightarrow A; C
    \end{gather*}
\end{task}

\begin{task}
    \begin{gather*}
        (A \to C) \to (\overline A \land B) \Rightarrow A \lor B \\
        (\overline A \lor C) \to (\overline A \land B) \Rightarrow A \lor B \\
        (\overline {\overline A \lor C}) \lor (\overline A \land B) \Rightarrow A \lor B \\
        (A \land \overline C) \lor (\overline A \land B) \Rightarrow A \lor B \\
        A \lor B \Rightarrow A \lor B
    \end{gather*}
\end{task}



