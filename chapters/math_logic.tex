\chapter{Элементы математической логики}
\section{Основные понятия и операции}
Пусть $A$ - множество и 
\begin{equation}
    a = 
    \begin{cases}
        1 &\text{$x \in A$}\\
        \emptyset &\text{$x \notin A$}
    \end{cases}
\end{equation}

Проведем аналогию из теории множеств в математическую логику:
\begin{gather}
    A \cup B \rightarrow a \lor b \\
    A \cap B \rightarrow a \land b \\
    \overline A \rightarrow \overline a \\
    A - B \rightarrow a - b = a \land \overline b \\
    A + b \rightarrow a + b = (a - b) \lor (b - a) = (a \land \overline b) \lor (b \land \overline a)
\end{gather}

\textbf{Дистрибутивность}
\begin{gather}
    a \lor (b \land c) = (a \lor b) \land (a \lor c) \\
    a \land (b \lor c) = (a \land b) \lor (a \land c)
\end{gather}

\textbf{Поглощение}
\begin{gather}
    a \lor (a \land b) = a \\
    a \land (a \lor b) = a
\end{gather}

\textbf{Законы де-Моргана}
\begin{gather}
    \overline{a \land b} = \overline a \lor \overline b \\
    \overline{a \lor b} = \overline a \land \overline b
\end{gather}

\textbf{Законы склеивания}
\begin{gather}
    (a \lor b) \land (a \lor \overline b) = a \\
    (a \land b) \lor (a \land \overline b) = a
\end{gather}

\textbf{Дополнительные логические операции}

\begin{enumerate}
    \item Стрелка Пирса:
        \begin{equation}
            a \downarrow b = \overline{a \lor b} = \overline a \land \overline b
        \end{equation}
    \item Штрих Шеффера:
        \begin{equation}
            a \mid b = \overline{a \land b} = \overline a \lor \overline b
        \end{equation}
    \item Импликация:
        \begin{equation}
            a \rightarrow b = \overline{a - b} = \overline{a \land \overline b} = \overline a \lor b
        \end{equation}
    \item Эквивалентность:
        \begin{equation}
            a \sim b = \overline{a + b} = \overline{(a \land \overline b) \lor (b \land \overline a)}
        \end{equation}
\end{enumerate}

\textbf{Совершенная дизъюнктивная нормальная функция (СДНФ)}
\begin{multline}
    f(x_1, x_2, ..., x_n) = (\overline x_1 \land \overline x_2 \land ... \land \overline x_n \land f(0, 0, ..., 0)) \\
    \lor (x_1 \land \overline x_2 \land ... \land \overline x_n \land f(1, 0, ..., 0)) \\
    \lor (x_1 \land x_2 \land ... \land x_n \land f(1, 1, ..., 1))
\end{multline}

\textbf{Совершенная конъюктивная нормальная функция (СКНФ)}
\begin{multline}
    f(x_1, x_2, ..., x_n) = (\overline x_1 \lor \overline x_2 \lor ... \lor \overline x_n \lor f(1, 1, ..., 1)) \\
    \land (x_1 \lor \overline x_2 \lor ... \lor \overline x_n \lor f(0, 1, ..., 1)) \\
    \land (x_1 \lor x_2 \lor ... \lor x_n \lor f(0, 0, ..., 0))
\end{multline}

\section{Многочлен Жегалкина}
Рассмотрим основные свойства

\begin{gather}
    \overline a = 1 + a \\
    a \land b = ab \\
    a \lor b = a + b + ab \\
    a + a = 0 \\
    a - b = a \land \overline b = a(1 + b) = a + ab \\
    a(b + c) = ab + ac \\
    a + b = b + a \\
    a \downarrow b = \overline{a \lor b} = \overline a \land \overline b = (1 + a)(1 + b) = 1 + a + b + ab \\
    a \mid b = \overline {a \land b} = 1 + ab \\
    a \rightarrow b = \overline{a - b} = \overline{a \land \overline b} = 1 + a(1 + b) = 1 + a + ab \\
    a \sim b = \overline{a + b} = 1 + a + b
\end{gather}

\textbf{Совершенная полиномиальная нормальная форма (СПНФ)}
Получается из СДНФ заменой $\lor$ на +, $\overline x = 1 + x$ и $x_1 \land x_2 = x_1x_2$
\begin{equation}
    (1 + a)(1 + b)f(0, 0) + a(1 + b)f(1, 0) + (1 + a)bf(0, 1) + abf(1, 1)
\end{equation}

\section{Задачи}

\begin{task}
    Проверить $(a \mid b) \mid c = a \mid (b \mid c)$
    \begin{solution}
        \begin{gather*}
            a \mid b = \overline {a \land b} = 1 + ab \\
            (a \mid b) \mid c = (1 + ab) \mid c = 1 + c(1 + ab) = 1 + c + abc \\
            a \mid (b \mid c) = a(1 + bc) = 1 + a(1 + bc) = 1 + a + abc
        \end{gather*}
        Ответ: неверное равенство.
    \end{solution}
\end{task}

\begin{task}
    Доказать $a \rightarrow (b \land c) = a \mid (b \mid c)$
    \begin{solution}
        \begin{equation*}
            a \rightarrow (b \land c) = \overline a \lor (\overline{\overline{b \land c}}) = \overline a \lor \overline{b \mid c} = a \mid (b \mid c)
        \end{equation*}
    \end{solution}
\end{task}

\begin{task}
    Доказать $a \downarrow ((b - a) \sim b) = 0$
    \begin{solution}
        \begin{multline*}
            a \downarrow (1 + (b - a) + b) = a \downarrow (1 + b + ab + b) = a \downarrow (1 + ab) = \\
            (1 + a)(1 + 1 + ab) = (1 + a)ab = \overline a ab = 0
        \end{multline*}
    \end{solution}
\end{task}

