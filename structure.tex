\usepackage[T2A]{fontenc} % Use 8-bit encoding that has 256 glyphs

\usepackage[utf8]{inputenc} % Required for including letters with accents

\usepackage[english,russian]{babel}
\usepackage{cmap}

\usepackage{indentfirst}
\setlength{\parindent}{0pt}

\usepackage{graphicx} % Required for including images
\graphicspath{{Figures/}} % Set the default folder for images

\usepackage{enumitem} % Required for manipulating the whitespace between and within lists

\usepackage{lipsum} % Used for inserting dummy 'Lorem ipsum' text into the template

\usepackage{subfig} % Required for creating figures with multiple parts (subfigures)

\usepackage{varioref} % More descriptive referencing

\usepackage{assoccnt}

\usepackage{xpatch}

\usepackage{amsmath,amssymb,amsthm,amsfonts} % For including math equations, theorems, symbols, etc

\usepackage{multicol}
\usepackage{graphicx}
\graphicspath{ {./images/} }

\theoremstyle{plain} % жирный заголовок, наклонный текст
\newtheorem{thm}{Теорема}
\newtheorem{lem}{Лемма}
\newtheorem{utv}[thm]{Утверждение} %счетчик утверждений иcпользует счетчик теорем - [thm] после объявления имени
\newtheorem{sle}{Следствие}[thm] %счетчик следствий подчинен счетчику теорем - [thm] в конце
\newtheorem*{sle*}{Следствие} %звездочка после newtheorem убирает нумерацию
\newtheorem*{remark*}{Замечение}

\theoremstyle{definition}
\newtheorem{task}{Задача}
\newtheorem*{solution}{Решение}
%----------------------------------------------------------------------------------------
%	HYPERLINKS
%---------------------------------------------------------------------------------------
\usepackage{hyperref}

\hypersetup{
    colorlinks=true,
    linkcolor=blue,
    filecolor=magenta,      
    urlcolor=cyan,
    pdftitle={Overleaf Example},
    pdfpagemode=FullScreen,
    }